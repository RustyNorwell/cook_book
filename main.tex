\documentclass[%
a4paper,
%twoside,
11pt
]{article}

% encoding, font, language
\usepackage[T1]{fontenc}
\usepackage[utf8]{inputenc}
\usepackage{lmodern}
%\usepackage{lm}
\usepackage[czech, english]{babel}

\usepackage{nicefrac}

\usepackage[
    nowarnings,
    %myconfig
]
{xcookybooky}

\usepackage{blindtext}    % only needed for generating test text

\DeclareRobustCommand{\textcelcius}{\ensuremath{^{\circ}\mathrm{C}}}


\setcounter{secnumdepth}{1}
\renewcommand*{\recipesection}[2][]
{%
    \subsection[#1]{#2}
}
\renewcommand{\subsectionmark}[1]
{% no implementation to display the section name instead
}


\usepackage{hyperref}    % must be the last package
\hypersetup{%
    pdfauthor            = {Sven Harder},
    pdftitle             = {Example Recipes for xcookybooky},
    pdfsubject           = {Recipes},
    pdfkeywords          = {example, recipes, cookbook, xcookybooky},
    pdfstartview         = {FitV},
    pdfview              = {FitH},
    pdfpagemode          = {UseNone}, % Options; UseNone, UseOutlines
    bookmarksopen        = {true},
    pdfpagetransition    = {Glitter},
    colorlinks           = {true},
    linkcolor            = {black},
    urlcolor             = {blue},
    citecolor            = {black},
    filecolor            = {black},
}

\setHeadlines {
	prephead = Postup,
	inghead = Ingredience,
	portionvalue = porce,
}

\setRecipeLengths {
	preparationwidth = 0.5 \textwidth ,
	ingredientswidth = 0.4 \textwidth ,
}

\hbadness=10000	% Ignore underfull boxes

\begin{document}

\title{Examples for using \textbf{xcookybooky}}
\author{Sven Harder\\ \href{mailto:sven\_one1@gmx.de}{sven\_one1@gmx.de}}
\maketitle

\begin{abstract}
    \noindent The examples in this document require at least version~1.4 of the \texttt{xcookybooky}\footnote{\url{http://www.ctan.org/pkg/xcookybooky}} package. For more examples and test recipes especially for using hook functions take a look at the source files located at \url{https://code.google.com/p/xcookybooky/}. If you are interested in modifying the layout of \texttt{xcookybooky} you will find examples in the documentation as well as in the configuration file \textbf{xcookybooky.cfg}.
\end{abstract}

\tableofcontents

\vspace{5em}

\section{Recepty}
The following recipes are examples for the usage of the \texttt{xcookybooky} package. The copyright of the pictures is owned by Roman Gaus. If you are using MiKTeX~2.9 you should get no errors, no warnings and no overfull boxes. The underfull boxes are suppressed due to the settings.

% background graphic
%\setBackgroundPicture[x, y=-2cm, width=\paperwidth-4cm, height, orientation = pagecenter]
%{pic/background}

%\include{tex/Mousse_au_Chocolat}

%\include{tex/Complete_Recipe}

%\include{tex/recipe}

\begin{recipe}
	[ %
	preparationtime = {\unit[20]{min}} ,
	bakingtime={\unit[30]{min}} ,
	%bakingtemperature={\protect\bakingtemperature{fanoven=\unit[230]{?C}}} ,
	%portion = {\portion[porce]{4}},
	portion = {\portion{4}},
	%calory={\unit[3]{kJ}} ,
	source = {Babička z Ostravy}
	]{Kuřecí prsa na kari}

	%description

	\ingredients{
		4 & kuřecí prsa\\
		3 & jablka\\
		8 lžiček & kari\\
		3 lžíce & sádla\\
		1 & cibule\\
		1 & smetana na šlehání\\
		3 lžíce & zakysané smetany\\
		& sůl\\
		& pepř\\
	}

	\preparation{
		\step Na rozehřátém sádle osmahněte na drobno nakrájenou cibuli a poté přidejte osolená opepřená a okariovaná kuřecí prsa.
		\step Po usmažení prsa proložte nastrouhanými jablky a nechte je 15-20
		minut dusit.
		\step Po rozměknutí maso vyjměte, rozmixujte jablečnou směs s výpekem, a zalijte šlehačkou a zakysanou smetanou.
		\step Směs nechte projít varem a dochuťte solí nebo pepřem.
	}
\end{recipe}


\begin{recipe}
	[ %
	preparationtime = {\unit[20]{min}} ,
	bakingtime={\unit[30]{min}} ,
	%bakingtemperature={\protect\bakingtemperature{fanoven=\unit[230]{?C}}} ,
	portion = {\portion[knedlíky]{3}},
	%calory={\unit[3]{kJ}} ,
	source = {Babička z Ostravy}
	]{Hrnkové knedlíky}

	%description

	\ingredients{
		1 hrnek & polotučného mléka\\
		1 hrnek & hrubé mouky\\
		1 & vejce\\
		1 & osmažený rohlík\\
		& máslo na vymazání\\
		& sůl\\
	}

	\preparation{
		\step Do misky nasypte 1 hrnek hrubé mouky, zalijte ji jedním hrnkem mléka, trochu osolte a nechte ustát.
		\step Do směsi přidejte jeden žloutek a jeden nakrájený osmažený rohlík.
		\step Z bílku z jednoho vejce ušlehejte sníh a smíchejte jej s těstem.
		\step Do vymazaných hrnků nalijte těsto. Nechte rezervu od okraje hrnku, ať knedlík nepřeteče.
		\step Hrnky dejte vařit do vody sahající nad spodní konec ucha hrnku a vařte 30-35 minut.
	}
\end{recipe}


\begin{recipe}
	[ %
	preparationtime = {\unit[80]{min}} ,
	bakingtime={\unit[12]{min}} ,
	bakingtemperature={\protect\bakingtemperature{topbottomheat=\unit[180]{\textcelsius}}},
	portion = {\portion[dávka]{1}},
	source = {\href{https://www.toprecepty.cz/recept/1155-vanocni-cukrovi-krehke-vanilkove-rohlicky/}{TopRecepty}}
	]{Vanilkové rohlíčky}

	\ingredients{
		340 g & másla\\
		100 g & cukru moučka\\
		420 g & hladké mouky\\
		160 g & mletých vlašských ořechů\\
		1 špetka & soli\\
		& moučkový cukr\\
		& vanilkový cukr\\
	}

	\preparation{
		\step Všechny suroviny na vále zpracujeme v pevné těsto, které necháme nejlépe přes noc v chladu odležet.
		\step Plech vyložíme pečicím papírem.
		\step Z těsta tvarujeme malé rohlíčky, ukládáme na plech a pečeme ve vyhřáté troubě na 180 °C.
		\step Ještě teplé rohlíčky obalujeme ve směsi moučkového cukru s vanilkovým cukrem.
	}
\end{recipe}


\end{document}
