\begin{recipe}
	[ %
	%preparationtime = {\unit[20]{min}} ,
	%bakingtime={\unit[30]{min}} ,
	%bakingtemperature={\protect\bakingtemperature{fanoven=\unit[230]{?C}}} ,
	portion = {\portion{4}},
	calory={\unit[8364]{kJ}} ,
	source = {\href{https://www.toprecepty.cz/recept/19960-cesnekove-kure-se-zampiony-a-smetanovo-vinnou-omackou/}{TopRecepty}}
	]{Česnekové kuře se žampiony a\\ smetanovo-vinnou omáčkou}

	%description

	\graph
	{% pictures
		small=images/cesnekove_kure_se_zampiony_na_smetane_2.png,
		big=images/cesnekove_kure_se_zampiony_na_smetane_1.png,
		smallpicturewidth = 0.5\textwidth,
		bigpicturewidth = 0.5\textwidth
	}

	\ingredients{
		3 & kuřecí prsní řízky\\
		250 g & žampionů\\
		1/2 paličky & česneku\\
		1 & cibule\\
		200 ml & bílého vína\\
		100 ml & silného vývaru\\
		200 ml & smetany ke šlehání\\
		& olej\\
		& máslo\\
		malá hrst & čerstvé hladkolisté petrželky\\
		& pepř\\
		& sůl\\
	}

	\preparation{
		\step Maso nakrájíme na větší kostky, osolíme, opepříme a necháme odležet.
		\step V hlubší pánvi rozpálíme olej a maso opečeme ze všech stran dočervena. Vyjmeme a udržujeme v teple, například v alobalu.
		\step Do výpečku přidáme kousek másla a když se rozehřeje, přisypeme na drobno nakrájenou cibuli a plátky žampionů. Lehce osolíme a opečeme. Případnou tekutinu prudce odpaříme. Přidáme víno, necháme ho zprudka odpařit, vlijeme vývar a vrátíme kousky masa.
		\step Přidáme česnek utřený se solí a dusíme asi 15 minut. Na závěr vlijeme smetanu, omáčku necháme zhoustnout nebo ji lehce povolíme vodou, vývarem či mlékem. Zjemníme kouskem másla a dochutíme čerstvou nasekanou petrželkou.
	}
\end{recipe}
